\documentclass[a4paper]{article}

\usepackage{afterpage}
\usepackage{graphicx}
\usepackage{subfigure}

% Define shorter ways to include individual images
\newcommand{\stufig}[4]						% images with default placement
{
	\begin{figure}
	\begin{center}
		\includegraphics[#1]{#2}
		\caption{#3}
		\label{#4}
	\end{center}
	\end{figure}
}

\newcommand{\stufigex}[5]					% images with specified placement
{
	\begin{figure}[#5]
	\begin{center}
		\includegraphics[#1]{#2}
		\caption{#3}
		\label{#4}
	\end{center}
	\end{figure}
}

% Define the stusubfig environment
\newenvironment{stusubfig}[1]
{
	\begin{figure}[#1]
	\begin{center}
}
{
	\end{center}
	\end{figure}
}

\begin{document}

\title{The Use of Zipping Algorithms to Facilitate Interaction with Image Partition Forests}
\author{Stuart Golodetz, Irina Voiculescu and Stephen Cameron}
\date{\today}
\maketitle

\begin{abstract}
\noindent TODO
\end{abstract}

\section{Introduction}

%TODO \cite{gvccimi08,gvcispa09}

Image segmentation, the problem of how to partition an image into regions that have some meaning in a given domain, and feature identification, the problem of how to ascribe meaning to some or all of those regions, are important tasks in a variety of contexts \cite{?}. However, as observed in \cite{golodetz11}, both remain challenging to completely automate due to the difficulty of specifying what constitutes a meaningful region in a particular context. In certain domains, particularly those such as medical imaging where the consequences of incorrect or inaccurate segmentation can be significant, it is therefore extremely important for domain specialists (e.g.~radiologists in the case of medical images) to be able to interact with the results of a segmentation produced by an automated tool in order to correct any mistakes made.

Depending on the kind of segmentation used\footnotemark, these results can be represented in a variety of different ways and require varying different methods of interaction. For example, editing a segmentation result produced by thresholding (a technique that classifies individual pixels of the image as either background or foreground based on their value relative to a constant threshold, e.g.~all pixels $<= 128$ are background and all pixels $> 128$ are foreground) might involve changing the threshold value and re-segmenting the image, whereas editing a segmentation produced by a snakes approach (e.g.~\cite{?}) might involve dragging the snake contour using a graphical user interface.

\footnotetext{A survey of many of the different kinds of segmentation can be found in \cite{golodetz11}.}

One type of segmentation result that can provide a great deal of useful information for further processing is what we call the \emph{image partition forest} (IPF) representation\footnotemark, which represents an image as a hierarchy of partitions (the details of this representation will be described in \S\ref{sec:ipfs}). This representation is useful because it consists of related segmentations of the image at different scales, providing a helpful space in which to search for regions corresponding to image features of different sizes. An IPF can be produced as the result of a number of different segmentation approaches, and detailed techniques for constructing one using watershed/waterfall segmentation are described in \cite{golodetz11}.

\footnotetext{The same data structure has appeared elsewhere in the literature under a variety of other names, e.g.~hierarchy of (region adjacency) graphs \cite{kropatsch04,nacken95,shen97}, hierarchical attributed region adjacency graph \cite{fischer04}, hierarchy of partitions \cite{haxhimusa03,lezoray06}, picture tree \cite{andrade03}, graph pyramid \cite{kerren06}, bounded irregular pyramid \cite{marfil07} and segmentation graph \cite{borenstein06}.}

However, despite the in many ways ubiquitous nature of the data structure itself in the literature, comparatively little research appears to have been done in the past into ways of conveniently editing an IPF after it has been constructed. (One notable exception to this is \cite{nacken95}, in which Nacken describes an approach to the problem of what he calls \emph{connectivity-preserving relinking} -- we tackled the same problem in a slightly more general way as \emph{parent switching}, or how to change the hierarchy so as to move a region from being the child of one parent to the child of another, as described in \cite{golodetz11}.) This is unfortunate, because it limits the applicability of techniques that automatically produce such results in domains such as medical imaging where the later introduction of expert knowledge is important. For that reason, this paper presents a number of the algorithms first described in \cite{golodetz11} for facilitating such editing, most notably multi-level split and merge algorithms known respectively as \emph{unzipping} and \emph{zipping}, and a \emph{non-sibling node merging} algorithm that allows the user to conveniently merge image-adjacent nodes in a GUI without worrying about the underlying structure of the IPF. A complete set of IPF algorithms, together with detailed implementations, can be found in \cite{golodetz11}.

The organisation of this paper is as follows: in \S\ref{sec:ipfs}, we provide a detailed description of the IPF data structure and outline the basic algorithms that can be performed upon it; in \S\ref{sec:zipping}, we present zipping algorithms for IPFs; in \S\ref{sec:nsmerge}, we present an algorithm for merging image-adjacent nodes; in \S\ref{sec:ui}, we demonstrate how our IPF algorithms can be used to facilitate user interaction by analysing a practical implementation of them in \emph{millipede}, a cross-platform tool for 3D segmentation, feature identification and visualisation; and in \S\ref{sec:conclusions}, we conclude.

\section{Image Partition Forests}
\label{sec:ipfs}

\subsection{Data Structure}

An image partition forest (or IPF) is a hierarchy of adjacency graphs that all partition the same image. As illustrated in Figure~\ref{fig:ipfs-concept}, which shows an IPF that might be constructed for a simple $4 \times 4$ image, an IPF consists of a number of layers. Every layer is an adjacency graph whose nodes correspond to contiguous regions of the underlying image. The regions $\{r_1,...,r_n\}$ corresponding to the nodes in each layer in an IPF over an image $I$ \emph{partition} $I$; that is, they satisfy $\bigcup_i r_i = I$ and $\forall i, j \cdot i \ne j \Rightarrow r_i \cap r_j = \emptyset$. Each layer is \emph{refined} by the next higher layer above it, in the sense that the region corresponding to each node in layer $i + 1$ is the union of some of the regions corresponding to nodes in layer $i$. For an example of this on a real image, see Figure~\ref{fig:ipfs-ctconcept}, which shows an IPF for a computerised tomography (CT) scan of the abdomen.

%---
\stufigex{height=19cm}{ipfs-concept.png}{The concept of a partition forest (see main text for discussion)}{fig:ipfs-concept}{p}
%---

%---
\stufigex{height=19cm}{ipfs-ctconcept.png}{An example partition forest for an abdominal CT scan}{fig:ipfs-ctconcept}{p}
%---

Each node in an IPF can have a set of properties associated with it, shown as blue text in Figure~\ref{fig:ipfs-concept}. Properties of nodes in the leaf layer (which correspond to individual pixels in the image) can be assigned arbitrarily, but properties of nodes in branch layers must be functions of the properties of their children in the layer below. In this example, a single, arbitrary value has been associated with each node in the leaf layer of the IPF. Nodes in higher layers have been given a `mean value' property that is calculated from the values of the subsumed leaf nodes.

Each layer also contains edges between nodes whose corresponding regions are adjacent in the image. Each edge has an associated value (shown as underlined text in the figure). The values on the edges in the leaf layer can be assigned according to any scheme desired -- in this example, they represent the height of the `lowest pass point' between adjacent regions, based on the values associated with the pixels. The value on an edge between a pair of nodes in a branch layer must be a function of the values on any edges between their respective children in the layer below. In this example, the value on an edge between two nodes, $u$ and $v$, in a branch layer is calculated to be the smallest value on any edge between a child of $u$ and a child of $v$, in keeping with the lowest pass idea above.

In addition to the layers themselves, an IPF also contains forest links that join the nodes in adjacent layers together (the coloured, dashed lines in the figure). In particular, there is a link between each node and the node that contains it in the layer above. These links naturally define parent/child relationships between forest nodes.

\subsection{Basic Algorithms}

As proved in \S4.3.1 of \cite{golodetz11}, an IPF for an image $I$ can be transformed into any other IPF for $I$ by the repeated application three basic algorithms: layer cloning, layer deletion and sibling node merging. For efficiency, we also add node splitting to this list, since it would be extremely inefficient (though possible) to implement node splitting in terms of the other three operations. Whilst full details of these algorithms (including how to implement them in a GUI environment) are given in \cite{golodetz11}, we nevertheless briefly describe them here because they provide a necessary background to the higher-level algorithms that follow.

\emph{Layer Cloning}. Any layer in the IPF can be cloned so as to insert a layer just above it with an identical adjacency graph. The forest links between the layer being cloned, the clone and any layer above the insertion point are updated accordingly. To operation can be undone by deleting the newly-inserted clone layer.

\emph{Layer Deletion}. Any layer in the IPF except the leaf layer can be deleted from the hierarchy. Any forest links that reference the deleted layer are removed, and new forest links are added where necessary between any layers on either side of the one being deleted. For the operation to be undoable, it is necessary to cache the layer when originally deleting it. The operation can then be undone by reinserting the layer into its original place in the hierarchy and recreating the forest links between it and the layers on either side of it (the information required to do this is present in the deleted layer).

\emph{Node Splitting}. Nodes in any branch layer of the IPF can be split into multiple nodes representing smaller contiguous image regions. The algorithm takes as input the node to be split and a set of groups of its children into which to split it. Intuitively, the node being split is removed, a new node is created to be the parent of each such group, and the adjacency graph and forest links are updated accordingly. The operation can be undone by performing a sibling node merge on the results of the split.

\emph{Sibling Node Merging}. Sibling nodes in any branch layer of the IPF can be merged provided that the union of the image regions they represent is a contiguous region (this preserves the invariant that every node in the IPF represents a contiguous region). A set of nodes are siblings if they either all have the same parent, or all have no parent (because they are in the top-most layer of the IPF). The details of how the algorithm can be implemented can be found in \cite{golodetz11}. To make the operation undoable, it is necessary to store the children of each of the nodes to be merged prior to performing the merge. The operation can then be undone using a node split, passing the child sets as input.

Sibling node merging is an important low-level operation, but it is inconvenient to use in the context of a user interface for IPF editing because it forces the user to think about the underlying hierarchical structure of the IPF. As the user of such a system, it is far more common to want to merge nodes that are adjacent in the image than to want to merge nodes that are adjacent in the IPF. This is the motivation behind the development of the far more general non-sibling node merging algorithm described in \S\ref{sec:nsmerge}.

\section{Unzipping and Zipping}
\label{sec:zipping}

TODO

\section{Non-Sibling Node Merging}
\label{sec:nsmerge}

TODO

\section{User Interaction}
\label{sec:ui}

TODO

\section{Conclusions}
\label{sec:conclusions}

TODO

\bibliographystyle{alpha}
\bibliography{existingwork,mypapers}

\end{document}
